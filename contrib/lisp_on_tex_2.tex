\documentclass[a4paper,landscape,columns=3]{cheatsheet}
\usepackage{framed}
\usepackage{pgfplots}
\usepackage{listings}
\usepackage{color}
\newcommand{\pT}[1]{\texttt{\textbackslash #1}}
%
% https://www.ctan.org/pkg/cheatsheet
%
\title{Lisp on \TeX \ II}
\author{by HAKUTA Shizuya \\
\\Kurt Pagani\\\href{mailto:nilqed@gmail.com}{nilqed@gmail.com}}
%\date{\today}
\date{August 8, 2022}
%
\lstdefinelanguage{LispOnTeX}
{keywords={define,defineM,lambda,setB,defmacro,macroexpand,let,letM,letrec,
           lispif,begin,callOCC,quote,cons,car,cdr,list,length,map,nth,
           writeList,equalQ,atomOrNilQ,append,subst,memberQ,pairlis,assoc,
           sublis,union,intersection,reverse,foldr,foldl,filter,allQ,anyQ},
keywords=[2]{concat,intTOstring,group,ungroup,expand,print,texprint,
             readLaTeXCounter,message,read,fgets},
keywords=[3]{symbolQ,stringQ,intQ,booleanQ,dimenQ,skipQ,pairQ,nilQ,funcQ,
             closureQ,macroQ,listQ,atomQ,procedureQ,isZeroQ,positiveQ,
             negativeQ},
keywords=[4]{mod,max,min,geq,leq,and,or,not},             
keywordstyle=[2]\color{red},
keywordstyle=[3]\color{teal},
keywordstyle=[4]\color{brown},
sensitive=true,%
alsoletter={\$},%
comment=[l]{\%},%
string=[b]",%
string=[b]'%
}

\lstset{frame=tb,
  language=LispOnTex,
  aboveskip=3mm,
  belowskip=3mm,
  showstringspaces=false,
  columns=flexible,
  basicstyle={\small\ttfamily},
  numbers=none,
  numberstyle=\tiny\color{gray},
  keywordstyle=\color{blue},
  commentstyle=\color{darkgray},
  stringstyle=\color{cyan},
  breaklines=true,
  breakatwhitespace=true,
  tabsize=3
}
%%%%%%%%%

%
\begin{document}
\maketitle
%
\textbf{Usage}: \pT{usepackage\{lisp-on-tex\}} \\
\pT{input\{listfun.sty\}}

\section{List functions}
\begin{lstlisting}
\writeList \lst \b \sep \e 
  % \lst the list to display
  % \b the begin character, e.g. '['
  % \sep the separator, e.g. ', '
  % \e the end character, e.g. ']'
Writes out a list in the form specified.
\end{lstlisting}
%
\begin{lstlisting}
\equalQ \x \y
  % \x, \y any lisp type
  % Tests equality recursively.
  % Note that atomQ () -> /f, i.e () is not an atom.
\end{lstlisting}
%
\begin{lstlisting}
\atomOrNilQ \x
  % Check whether \x is an atom or the empty list (). 
  % Returns /f otherwise.
\end{lstlisting}
%
\begin{lstlisting}
\append \x \y
  % Append \y to the list \x.
\end{lstlisting}
%
\begin{lstlisting}
\subst \x \y \z
  % Substitute \x for \y in the list \z.
\end{lstlisting}
%
\begin{lstlisting}
\memberQ \x \y
  % If \x is a member of \y then return /t else /f. 
  % Note: \x may be a sublist, and atoms are members only on first level!
\end{lstlisting}
%
\begin{lstlisting}
\pairlis \x \y \a
  % Give the list of pairs of corresponding elements of the lists \x and
  % \y, and appends this to the list \a. The resultant list of pairs, which 
  % is like a table with two columns, is called an association list. 
\end{lstlisting}
%
\begin{lstlisting}
\assoc \x \a
  % If \a is an association list, then \assoc will produce the first pair 
  % whose first term is \x. Thus it is a table searching function. 
\end{lstlisting}
%
\begin{lstlisting}
\sublis \a \y 
  % Here \a is assumed to be an association list of the form 
  % ((ul . v l ) . . . (un . v,)), where the u1's are atomic, and \y is 
  % any S-expression. What \sublis does, is to treat the u1's as variables 
  % when they occur in \y, and to substitute the corresponding v1's
  % from the pair list. 
  % Note: \sublisXXX is the helper function sub2 in the LISP 1.5 Programmer
  %  Manual (from where we have the info:).
\end{lstlisting}
%
\begin{lstlisting}
\union \x \y 
  % Union of the lists \x and \y
\end{lstlisting}
%
\begin{lstlisting}
\intersection \x \y 
  % Intersection of the lists \x and \y
\end{lstlisting}
%
\begin{lstlisting}
\reverse \x 
  % Reverse the list \x
\end{lstlisting}
%
\begin{lstlisting}
\foldr \f \x \y 
  % Fold right list \y with \f and start \x.  
\end{lstlisting}
%
\begin{lstlisting}
\foldl \f \x \y 
  % Fold left list \y with \f and start \x.  
\end{lstlisting}
%
\begin{lstlisting}
\filter \f \x 
  % Filter the list with the function \f
\end{lstlisting}
%
\begin{lstlisting}
\allQ \f \x 
  % f(x) true for all x?
\end{lstlisting}
%
\begin{lstlisting}
\anyQ \f \x 
  % f(x) true for any x?
\end{lstlisting}
%


\scriptsize{For details and examples consult the manuals} \\
\scriptsize{\url{https://github.com/nilqed/lisp-on-tex}}

\end{document}