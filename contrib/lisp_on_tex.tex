\documentclass[a4paper,landscape,columns=3]{cheatsheet}
\usepackage{framed}
\usepackage{pgfplots}
\usepackage{listings}
\usepackage{color}
\newcommand{\pT}[1]{\texttt{\textbackslash #1}}
%
% https://www.ctan.org/pkg/cheatsheet
%
\title{Lisp on \TeX}
\author{by HAKUTA Shizuya \\
\\Kurt Pagani\\\href{mailto:nilqed@gmail.com}{nilqed@gmail.com}}
%\date{\today}
\date{August 8, 2022}
%
\lstdefinelanguage{LispOnTeX}
{keywords={define,defineM,lambda,setB,defmacro,macroexpand,let,letM,letrec,
           lispif,begin,callOCC,quote,cons,car,cdr,list,length,map,nth},
keywords=[2]{concat,intTOstring,group,ungroup,expand,print,texprint,
             readLaTeXCounter,message,read,fgets},
keywords=[3]{symbolQ,stringQ,intQ,booleanQ,dimenQ,skipQ,pairQ,nilQ,funcQ,
             closureQ,macroQ,listQ,atomQ,procedureQ,isZeroQ,positiveQ,
             negativeQ},
keywords=[4]{mod,max,min,geq,leq,and,or,not},             
keywordstyle=[2]\color{red},
keywordstyle=[3]\color{teal},
keywordstyle=[4]\color{brown},
sensitive=true,%
alsoletter={\$},%
comment=[l]{//},%
string=[b]",%
string=[b]'%
}

\lstset{frame=tb,
  language=LispOnTex,
  aboveskip=3mm,
  belowskip=3mm,
  showstringspaces=false,
  columns=flexible,
  basicstyle={\small\ttfamily},
  numbers=none,
  numberstyle=\tiny\color{gray},
  keywordstyle=\color{blue},
  commentstyle=\color{dkgreen},
  stringstyle=\color{cyan},
  breaklines=true,
  breakatwhitespace=true,
  tabsize=3
}
%%%%%%%%%

%
\begin{document}
\maketitle
%
\textbf{Usage}: \pT{usepackage\{lisp-on-tex\}}

\section{Syntax}
\resizebox{9cm}{!}{
\begin{tabular}{|l|l}
\hline
Kinds  & Literals \\        
\hline\hline
\tt{CONS Cell} & \tt{`(` *obj* ... `.` *obj* `)`, `(` *obj* ... `)`} \\
\tt{Integer}   & \tt{`:` *TeX's integer*}                            \\
\tt{String}    & \tt{`'` *TeX's balanced tokens* `'`}                \\
\tt{Symbol}    & \tt{*TeX's control sequence*}                       \\
\tt{Boolean}   & \tt{`/t` or `/f`}                                   \\
\tt{Nil}       & \tt{`()`}                                           \\
\tt{Skip}      & \tt{`@` *TeX's skip*}                               \\
\tt{Dimen}     & \tt{`!` *TeX's dimen*}                              \\
\hline
\end{tabular}}
%
\section{Definition}
\begin{lstlisting}
  \define  : Define a symbol.
  \defineM : Define a mutable symbol.
  \setB : Rewrite a mutable symbol.
  \defmacro : Define a macro.
  \macroexpand : Expand a macro
  \lambda : Create a function.
  \let : Define local symbols.
  \letM : Define mutable local symbols.
  \letrec : Define local symbols recursively.
\end{lstlisting}
%
\section{Control Flow}
\begin{lstlisting}
  \lispif : Branch.
  \begin : Execute expressions.
  \callOCC : One-shot continuation.
\end{lstlisting}
%
%
\section{String Manipulations}
\begin{lstlisting}
  \concat : Concatenate tokens.
  \intTOstring : Convert a integer to TeX tokens.
  \group : Grouping.
  \ungroup : Ungrouping.
  \expand : Expand tokens.
\end{lstlisting}
%
%
\section{Arithmetical Functions}
\begin{lstlisting}
  \+ : Addition.
  \- : Subtraction.
  \* : Multiplication.
  \/ : Division.
  \mod : Modulo.
  \>, \<, \geq, \leq : Comparison.
  \max : Maximum.
  \min : Minimum.
\end{lstlisting}
%
%
\section{Logical functions}
\begin{lstlisting}
  \and, \or, \not : Logical and, or, not 
\end{lstlisting}
%
%
\section{Traditional LISP Functions and Special Forms}
\begin{lstlisting}
  \quote : Quote.
  \cons, \car, \cdr : CONS, CAR, CDR
  \list : Create a list
  \length : Get the size of a LIST.
  \map : Map function.
  \nth : Get the n-th value of a LIST (starting with 0).
  \= : Equality.
  \texprint : Convert a object to TeX tokens and output it to the document
  \print : (For test) output a object as TeX tokens
\end{lstlisting}
%
%
\section{Type Predicates}
\begin{lstlisting}
  (\symbolQ (\quote \cs))
  (\stringQ 'foo')
  (\intQ :42)
  (\booleanQ /f)
  (\dimenQ !12pt)
  (\skipQ @12pt plus 1in minus 3mm)
  (\pairQ (\cons :1 :2))
  (\nilQ ())
  (\funcQ \+)
  (\closureQ (\lambda () ()))
  (\defmacro (\x) ())
  (\macroQ \x)
  (\listQ ())
  (\listQ (\list :1 :2))
  (\atomQ :23)
  (\atomQ 'bar')
  (\procedureQ \+)
  (\procedureQ (\lambda () ()))
  (\isZeroQ :0)    % /t
  (\positiveQ :42) % /t
  (\negativeQ :-2) % /t
\end{lstlisting}
%
%
\section{\LaTeX Utils}
\begin{lstlisting}
  \readLaTeXCounter : Read an integer from LaTeX
  \message : Wrapper of LaTeX \message
  \read : Read a LISP expression from stdin
  \fgets : Read a string from stdin.
\end{lstlisting}

\subsection{Class Options}
\begin{tabular}{|c|c|}
\hline 
Option Name & Meaning \\ 
\hline\hline 
\tt{noGC} & Never use GC (default) \\ 
\hline 
\tt{markGC} & Using Mark-Sweep GC \\ 
\hline 
\tt{GCopt=...} & Passing option to the GC engine \\ 
\hline 
\end{tabular} 
%
%
\section{Additional Packages}
\subsection{Fixed Point Numbers}
The package \texttt{lisp-mod-fpnum} adds fixed point numbers to LISP on \TeX. 
Load it by \pT{usepackage}:

\begin{verbatim}
\usepackage{lisp-on-tex}
\usepackage{lisp-mod-fpnum}
\end{verbatim}
%
\subsection{Regular Expressions}
The package \texttt{lisp-mod-l3regex} is thin wrapper of 
\texttt{l3regex}. Load it by \pT{usepackage}:

\begin{verbatim}
  \usepackage{lisp-on-tex}
  \usepackage{lisp-mod-l3regex}
\end{verbatim}

\scriptsize{For details and examples consult the manual.} \\
\scriptsize{\url{https://github.com/nilqed/lisp-on-tex}}
\end{document}